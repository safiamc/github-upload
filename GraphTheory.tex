\documentclass{beamer}
\mode<presentation>
{
  \usetheme{Warsaw}
  % or ...

  \setbeamercovered{transparent}
  % or whatever (possibly just delete it)
}


\usepackage[english]{babel}
\usepackage{amsmath}
\usepackage{amssymb}
\usepackage{mathrsfs}
\usepackage{amsthm}
\usepackage{mathtools}
\usepackage{thmtools}
\usepackage{listings}
\usepackage{xcolor}
\usepackage{rotating}
\usepackage[utf8]{inputenc}
\usepackage{animate}
\usepackage{centernot}
\usepackage{tikz-cd}
\usepackage{times}
%\usepackage[T1]{fontenc}

\DeclareMathOperator{\Conf}{Conf}
\DeclareMathOperator{\UConf}{\overline{Conf}}
\DeclareMathOperator{\Star}{Star}
\newcommand{\R}{\mathbb{R}}
\newcommand{\N}{\mathbb{N}}
\newcommand{\C}{\Conf_n(X)}
\newcommand{\G}{\Gamma}
\newcommand{\CG}{\Conf_n(\Gamma)}
\newcommand{\Z}{\mathbb{Z}}
\newcommand{\UC}{\UConf_n(X)}
\newcommand{\UCG}{\UConf_n(\Gamma)}
\newcommand{\DG}{\mathcal{D}_n(\Gamma)}
\newcommand{\KG}{K_n(\Gamma)}
\newcommand{\DKG}{\mathcal{D}_{n,k}(\Gamma)}
\newcommand{\CKG}{\Conf_{n,k}(\Gamma)}
\newtheorem{proposition}[theorem]{Proposition}
\newtheorem{claim}[theorem]{Claim}
\newtheorem{conj}[theorem]{Conjecture}
\title{Impossible Walks and Planar Graphs} 

%\subtitle
%{Include Only If Paper Has a Subtitle}

\author
{Safia Chettih}
% - Give the names in the same order as the appear in the paper.
% - Use the \inst{?} command only if the authors have different
%   affiliation.

\institute{Reed College} % (optional, but mostly needed)


\date[ATS] 
{St Mary's Academy, April 24th 2018}



% If you have a file called "university-logo-filename.xxx", where xxx
% is a graphic format that can be processed by latex or pdflatex,
% resp., then you can add a logo as follows:

 \pgfdeclareimage[height=0.5cm]{university-logo}{griffin.png}
 \logo{\pgfuseimage{university-logo}}

% If you wish to uncover everything in a step-wise fashion, uncomment
% the following command: 

%\beamerdefaultoverlayspecification{<+->}

\begin{document}
\begin{frame}
\titlepage
\end{frame}

\section{Eulerian Walks}

\begin{frame}{Bridges of Königsberg}
\centering
At first, there were 6 bridges in the town of Königsberg\\
\includegraphics[scale=0.5]{6bridge.png}

and you could take a stroll across each bridge exactly once.
\end{frame}
\begin{frame}{Bridges of Königsberg}
\centering
At first, there were 6 bridges in the town of Königsberg\\
\includegraphics[scale=0.5]{6bridgep.png}

and you could take a stroll across each bridge exactly once.
\end{frame}
\begin{frame}{Bridges of Königsberg}
\centering
But when they built another bridge, no one could figure out how to do it anymore.\\
\includegraphics[scale=0.5]{7bridge.png}

It seemed like no matter which way they went, they would always miss a bridge or get stuck.
\end{frame}
\begin{frame}{Bridges of Königsberg}
\centering
But when they built another bridge, no one could figure out how to do it anymore.\\
\includegraphics[scale=0.5]{7bridgep1.png}

It seemed like no matter which way they went, they would always miss a bridge or get stuck.
\end{frame}\begin{frame}{Bridges of Königsberg}
\centering
But when they built another bridge, no one could figure out how to do it anymore.\\
\includegraphics[scale=0.5]{7bridgep2.png}

It seemed like no matter which way they went, they would always miss a bridge or get stuck.
\end{frame}
\begin{frame}{Bridges of Königsberg}
\centering
Does a walk that passes through each bridge exactly once exist?\\
\includegraphics[scale=0.5]{7bridge.png}
\end{frame}

\begin{frame}{Bridges of Königsberg}
\centering
Does a walk that passes through each bridge exactly once exist?\\
\includegraphics[scale=0.5]{7bridgeg1.png}
\end{frame}

\begin{frame}{Bridges of Königsberg}
\centering
Does a walk that passes through each bridge exactly once exist?\\
\includegraphics[scale=0.5]{7bridgeg2.png}
\end{frame}
\begin{frame}{Graphs}
Is there a walk on the graph that passes through each edge exactly once?\\
\centering
\includegraphics[scale=0.5]{Euler_graph}
\end{frame}
\begin{frame}{Graphs}
Is there a walk on the graph that passes through each edge exactly once?\\
NO \hfill
\includegraphics[scale=0.25]{Euler_graph}\hfill
\includegraphics[scale=0.2]{bball_graph}\hfill
\includegraphics[scale=0.2]{pent_graph}\\
\hrulefill\vspace*{.2cm}\\
YES \hfill
\includegraphics[scale=0.25]{Euler_graphno}\hfill
\includegraphics[scale=0.2]{bball_graphno}\hfill
\includegraphics[scale=0.2]{pent_graphno}

\end{frame}
\begin{frame}{Definitions}
A \textbf{graph} is a collection of vertices and edges between them.\\
How we draw a graph is not unique.\\
\includegraphics[scale=0.4]{K4}
\includegraphics[scale=0.4]{K4_planar}
\end{frame}
\begin{frame}{Definitions}
The \textbf{degree} of a vertex is the number of edges attached to it.

\begin{minipage}{.5\textwidth}
\hspace*{.5cm}
\includegraphics[scale=0.25]{bball_graph}
\end{minipage}
\begin{minipage}{.4\textwidth}
Every vertex in this graph has degree 3.
\end{minipage}

\begin{minipage}{.5\textwidth}
\hspace*{-.5cm}
\includegraphics[scale=0.3]{Euler_graphno}
\end{minipage}
\begin{minipage}{.4\textwidth}
The degrees of the vertices in this graph are 4, 3, 3, 2.
\end{minipage}
\end{frame}
\begin{frame}
How does degree relate to the number of edges?\vspace*{1cm}\\
\begin{minipage}{.3\textwidth}
\centering
\includegraphics[scale=0.2]{bball_graph}\\
deg: 3, 3, 3, 3\\
\# of edges: 6
\end{minipage}
\begin{minipage}{.3\textwidth}
\centering
\includegraphics[scale=0.25]{Euler_graphno}\\
deg: 4, 3, 3, 2\\
\# of edges: 6
\end{minipage}
\begin{minipage}{.3\textwidth}
\centering
\includegraphics[scale=0.25]{Euler_graph}\\
deg: 5, 3, 3, 3\\
\# of edges: 7
\end{minipage}
\end{frame}
\begin{frame}{Multiple Choice}
Suppose I have a graph whose vertices have degrees 2, 2, 1, 1. How many edges does it have?
\begin{itemize}
\item 2
\item 3
\item 4
\item 5
\end{itemize}
\end{frame}
\begin{frame}{Multiple Choice}
Suppose I have a graph whose vertices have degrees 2, 2, 1, 1. How many edges does it have?
\begin{columns}
\begin{column}{.5\textwidth}
\begin{itemize}
\item 2
\item \fcolorbox{red}{white}{3}
\item 4
\item 5
\end{itemize}
\end{column}
\begin{column}{.5\textwidth}
\includegraphics[scale=0.4]{mchoice1}
\end{column}
\end{columns}
Why? If we sum all the degrees, each edge gets counted twice.
\end{frame}
\begin{frame}{Degree Theorem}
\begin{theorem} The sum of the degrees of all vertices in a graph is twice the number of edges.
$$ \sum_{v \in V(G)} \deg{v} = 2E$$
\end{theorem}
\textit{Corollary.} The number of vertices with odd degree is even.\medskip\\
It is impossible to draw a graph with an odd number of vertices where all of them have odd degree!
\end{frame}
\begin{frame}{Definitions}
Two vertices are \textbf{adjacent} if there is an edge between them.\\
A \textbf{walk} is a sequence of adjacent vertices and the edges between them.\vspace*{.5cm}\\
{\centering \includegraphics[scale=0.5]{K4_walk}}
\end{frame}
\begin{frame}
How do walks relate to degree?\vspace{.5cm}
\begin{columns}
\begin{column}{.5\textwidth}
If a walk enters a vertex with even degree, it exits as well.\vspace*{.2cm}\\
\includegraphics[scale=0.3]{even_deg}
\end{column}
\begin{column}{.5\textwidth}
If a walk enters a vertex with odd degree, it may get stuck or it may exit.\vspace*{.2cm}\\
\includegraphics[scale=0.3]{odd_deg}
\end{column}
\end{columns}\vspace*{.2cm}
Walks that pass through vertices leave untouched edges with the same parity (even or odd) as the degree of the vertex.
\end{frame}
\begin{frame}{Eulerian Walks} A walk is \textbf{Eulerian} (pronounced ``oilerian'') if it passes through each edge of the graph exactly once.
\begin{description}
\item[4 or more vertices of odd degree:] there is no Eulerian walk\\
\includegraphics[scale=0.2]{K4_planar}
\item[exactly 3 vertice of odd degree:] impossible by corollary, because vertices with odd degree occur in pairs

\end{description}
\end{frame}
\begin{frame}{Eulerian Walks}
\begin{description}
\item[exactly 2 vertices of odd degree:] any Eulerian walk starts at one and ends at the other\\
\includegraphics[scale=0.2]{Euler_graphno}
\item[exactly 1 vertex of odd degree:] impossible by corollary
\item[no vertices of odd degree:] any Eulerian walk must start and end in the same place\\
\includegraphics[scale=0.2]{pent_graphno}
\end{description}
\end{frame}
\begin{frame}{Finding Eulerian Walks}
Look for the vertices of odd degree!\vspace*{.2cm}\\
\includegraphics[scale=0.5]{Euler_ex}
\end{frame}
\begin{frame}{Finding Eulerian Walks}
Look for the vertices of odd degree!\vspace*{.2cm}\\
\includegraphics[scale=0.5]{Euler_ex_walk}
\end{frame}
\section{Planar Graphs}
\begin{frame}{Definitions}
A \textbf{connected} graph has at least one walk between any pair of vertices.\\
If we draw a connected graph so that no edges cross, we can count the number of faces of the graph by counting the `holes', including the empty space around the graph.\vspace*{.2cm}\\
{\centering
\includegraphics[scale=0.3]{K4_formula}\\
\hfill $F=4$, $E=6$, $V=4$\hfill}
\end{frame}
\begin{frame}{Euler's Formula}
\begin{columns}
\begin{column}{0.35\textwidth}
\centering
\includegraphics[scale=0.2]{pent_graph}\\
$F=5$\\ $E=8$\\ $V=5$
\end{column}
\begin{column}{0.35\textwidth}
\centering
\includegraphics[scale=0.2]{loop_graph}\\
$F=3$\\ $E=5$\\$V=4$
\end{column}
\begin{column}{0.35\textwidth}
\centering
\includegraphics[scale=0.2]{bball_graph}\\
$F=4$\\ $E=6$\\ $V=4$
\end{column}
\end{columns}
\end{frame}
\begin{frame}{Multiple Choice}
Suppose I have a connected graph with 3 faces and 3 vertices. How many edges does the graph have, if it can be drawn so that none of the edges are crossing?
\begin{itemize}
\item 3
\item 4
\item 5
\item 6
\end{itemize}
\end{frame}
\begin{frame}{Multiple Choice}
Suppose I have a connected graph with 3 faces and 3 vertices. How many edges does the graph have, if it can be drawn so that none of the edges are crossing?
\begin{columns}
\begin{column}{.5\textwidth}
\begin{itemize}
\item 3
\item  \fcolorbox{red}{white}{4}
\item 5
\item 6
\end{itemize}
\end{column}
\begin{column}{.5\textwidth}
\includegraphics[scale=0.4]{mchoice2}
\end{column}
\end{columns}
$F+V=E+2$. Why?
\end{frame}
\begin{frame}{Proof by Induction}
If we have the most simple graph possible, we can check that it satisfies $F+V=E+2$:\vspace*{.2cm}\\
\includegraphics[scale=0.2]{bar}\vspace*{.2cm}\\
If we add a vertex, we have to connect it with an edge, $F+(V+1)=(E+1)+2$:\vspace*{.2cm}\\
\includegraphics[scale=0.2]{vertex_add}
\end{frame}
\begin{frame}{Euler's Formula}
If we add an edge between two existing vertices, we divide a face into two, $(F+1)+V=(E+1)+2$:\vspace*{.2cm}\\
\includegraphics[scale=0.2]{edge_add}\vspace*{.2cm}\\
A \textbf{planar} graph is a graph that can be drawn so that there are no edges crossing.
\begin{theorem}[Euler] For any connected, planar graph, $F+V=E+2$.
\end{theorem}
\end{frame}
\begin{frame}{Nonplanar Graphs}
Not every graph is planar!\vspace*{.2cm}\\
\includegraphics[scale=0.5]{K33} \hfill \includegraphics[scale=0.4]{K33_alt}\vspace*{.2cm}\\
This graph violates Euler's Formula.
\end{frame}

\begin{frame}{Nonplanar Graphs}
\includegraphics[scale=0.25]{K33_alt}\vspace*{.2cm}\\
Any face must have at least 4 edges on its boundary, but each edge is counted twice, so $E\geq\frac{4F}{2}=2F$.\\
 But $V=6$, $E=9$ so Euler's formula gives 
 \begin{align*}
 F+6 &= 9 + 2 = 11\\
 F=5
 \end{align*}
 But $E = 9 \not\geq2F= 2 \cdot 5 = 10$, so the graph can't possibly be planar.
\end{frame}
\begin{frame}{Summary}
\begin{itemize}
\item To know whether a graph has a walk that passes through each edge exactly once, count the number of vertices with an odd number of edges coming out. If there are 4 or more, then it's impossible.
\item If you can draw a graph with no edges crossing, then it satisfies the formula $F+V=E+2$.
\item If it doesn't satisfy the formula, then every drawing will have at least one crossing.
\end{itemize}
\end{frame}
\begin{frame}
\huge{Thank you!}\vspace*{.2cm}
    \normalsize
    
  My email:
  \texttt{safia@reed.edu}
  \begin{thebibliography}{10}

  \beamertemplatearticlebibitems
  \bibitem{Trudeau}
    R.~Trudeau.
\newblock Introduction to Graph Theory (Dover Books on Mathematics).
\bibitem{Fasc}
 A.~Benjamin, G.~ Chartrand, and P.~Zhang.
 \newblock The Fascinating World of Graph Theory.
 \bibitem{brill}
  \url{https://brilliant.org/wiki/graph-theory/}
   \end{thebibliography}
  
\end{frame}

\end{document}
